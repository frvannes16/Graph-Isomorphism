\documentclass[11pt]{article}
\usepackage{cite}

\begin{document}

\title{Graph Isomorphism}
\author{Franklin van Nes}
\date{Tuesday 12th November, 2017}
\maketitle

Graph Isomorphism is very much defined in its name: Iso (same) morphism (shape) of graphs.
In more detail, two finite graphs are isomorphic if they share the same number of vertices connected in the same way.
Formally, two finite graphs $G$ and $H$ with graph vertices $G.V$ and $H.V$ labelled $V_n = \{1, 2, 3, 4, ... , n\}$, and graph edges $G.E$ and $H.E$
are said to be isomorphic if $f: G.V \rightarrow H.V$ is an edge-preserving bijection. Meaning
that there exists a permutation $p$ of $V_n$ so that edges $(u, v) \in G.E \iff (p(u), p(v)) \in E.V$ ~\cite{gary-definer}.

\bibliography{mybib}{}
\bibliographystyle{abbrv}
\end{document}
